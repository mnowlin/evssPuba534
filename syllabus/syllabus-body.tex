\hypertarget{covid-19}{%
\section{COVID-19}\label{covid-19}}

The COVID-19 pandemic is still ongoing. \textbf{The College of
Charleston requires that masks be worn while indoors and you must wear a
mask at all times while in class.} Although vaccinations are currently
not required, \emph{I ask you to be respectful of the health and safety
of others}. If you have not received the \textbf{COVID-19 vaccine (as
well as the booster), which is safe, free, and effective, please
consider doing so immediately}. Information about the vaccine is
available from the
\href{https://scdhec.gov/covid19/covid-19-vaccine}{SCDHEC website} and
information about where and when to obtain a vaccine is also available
on the SCDHEC website \href{https://vaxlocator.dhec.sc.gov/}{vaccine
locator page}.

\hypertarget{course-description}{%
\section{Course Description}\label{course-description}}

Environmental law in the United States is a complex thicket of
legislation, agency regulations, and court decisions that shapes how the
US approaches environmental issues. This course will provide an overview
of several major topics within environmental law as well as the
regulatory instruments used to address environmental problems. The aim
of this course is for students in the MES, MPA, or concurrent MES/MPA
programs to gain an understanding of the development and scope of
environmental law in the US.

\vspace{0.1in}

\noindent The course is divided into four sections, 1)
\textbf{Foundations}: includes some of the foundational aspects of
environmental law including environmental risks, enforcement, key
actors, sources of environmental laws, administrative law, and
rulemaking; 2) \textbf{Pollution}: covers legal and regulatory
approaches for air pollution, climate change, water pollution, drinking
water, toxic substances, and waste management; 3) \textbf{Ecosystem
Services}: discusses management of natural resources and includes the
public trust doctrine, wetlands, land use, biodiversity, and endangered
species; and 4) \textbf{Environmental Impact Statements}: examines the
National Environmental Policy Act (NEPA).

\vspace{0.1in}

\noindent Laptops are allowed, but should only be used to access the
readings and for in-class group assignments. Phones should be put away
during class. \emph{I encourage you to take notes by hand, with pen and
paper}.
\href{https://www.nytimes.com/2017/11/27/learning/should-teachers-and-professors-ban-student-use-of-laptops-in-class.html}{You
learn better that way}. I recommend taking notes using the
\href{http://www.usu.edu/arc/idea_sheets/pdf/note_taking_cornell.pdf}{Cornell
Method}.

\hypertarget{naspaa-competencies-and-course-learning-objectives}{%
\subsection{NASPAA Competencies and Course Learning
Objectives}\label{naspaa-competencies-and-course-learning-objectives}}

Students who graduate from NASPAA accredited MPA programs should develop
the ability to: lead and manage in public governance, participate in and
contribute to the policy process, apply skills in analysis and critical
thinking to solve problems and make decisions, articulate and apply a
public service perspective, and communicate and interact productively
with a diverse and changing workforce and citizenry. \textbf{This course
is designed with a special emphasis on applying skills in analysis and
critical thinking to solve problems and make decisions}. To that end,
learning objectives include:

\begin{itemize}
\item
  Develop an understanding of key environmental statues
\item
  Evaluate the various regulatory instruments used to address
  environmental problems
\item
  Analyze a specific environmental problem and the legal and regulatory
  approaches used to address it
\item
  Display oral, written, and group communication skills
\end{itemize}

\vspace{0.10in}

\noindent These objectives will be achieved through critically reading
the course readings; by writing several short reflection papers; and by
completing a paper about a specific environmental regulation.

\hypertarget{required-readings}{%
\section{Required Readings}\label{required-readings}}

There are no required books to purchase. All the readings will be made
available on \href{https://lms.cofc.edu}{OAKS} for each week. Readings
should be read prior to class and you should come to class prepared to
discuss the readings.

\hypertarget{attendance-policy}{%
\subsection{Attendance Policy}\label{attendance-policy}}

Attendance will be taken for each class session, and will be part of
your course engagement grade. You are allowed to miss \emph{one class
without penalty}. \textbf{However, do not come to class if you feel
ill.~Additionally, if you have been exposed to or tested positive for
COVID-19, do not come to class regardless of how you feel. In those
cases, I am happy to meet with you on Zoom to discuss material you
missed and wave the attendance requirement. Just let me know.}

\hypertarget{course-requirements-and-grading}{%
\section{Course Requirements and
Grading}\label{course-requirements-and-grading}}

Performance in this course will be evaluated on the basis of 10
reflection papers, a paper, a presentation, in-class group assignments,
and course engagement including attendance. \emph{Instructions for each
assignment will be placed on OAKS}. Due dates are in the schedule below.
Points will be distributed as follows:

\vspace{0.10in}
\begin{tabular}{ l l}
\hline
Assignment & Possible Points \\
\hline 
Reflection Papers (10 at 20 points each) & 200 points total \\
Paper & 150 points \\
Presentation & 50 points \\
In-class Assignments & 50 points \\
Engagement & 50 points \\ 
\hline
Total & 500 points \\
\hline
\end{tabular}

\vspace{0.10in}

\noindent \emph{There are 500 possible points for this course. Grades
will be allocated based on your earned points and calculated as a
percentage of 500}: A = 90 to 100\%; B+ = 87 to 89\%; B = 80 to 86\%; C+
= 77 to 79\%; C = 70 to 76\%; F = 69\% and below

\hypertarget{assignments}{%
\subsection{Assignments}\label{assignments}}

\textbf{Specific instructions for the following assignments are posted
on \href{https://lms.cofc.edu/d2l/home}{OAKS}. All work must be turned
in through the Assignment folder on OAKS, and is due at class time:
Thursday, 5:30 PM Eastern}.

\begin{itemize}
\item
  \emph{Reflection Papers}: You will write 10 short, about 2-3 pages,
  reflection papers that summarize and integrate the readings. Prompts
  will be given for each paper in \href{https://lms.cofc.edu}{OAKS}.
  Note that 12 are assigned, but only 10 will be graded. \textbf{When
  assigned, reflection papers are due at class time.}
\item
  \emph{Paper}: You will write a 10-12 page paper that covers a major
  state or federal environmental regulation. In the paper you will
  describe the regulation, discuss the environmental problem it is meant
  to address, discuss the legislative authority for the regulation, and
  court decisions that are relevant to the regulation. \textbf{The paper
  is due on the last day of classes, April 21, at class time.}
\end{itemize}

\begin{itemize}
\item
  \emph{Presentation}: You will give a short, 5 to 7 minute,
  presentation about the regulation you wrote about for your paper.
  \textbf{Presentations will be given on the last day of classes, April
  21.}
\item
  \emph{In-class Assignments}: There will be several in-class
  assignments, both individual and group that will be given through-out
  the semester. You need to in-class to receive credit for the in-class
  assignment. \textbf{However, do not come to class if you have been
  exposed to or tested positive for COVID. In that case, I will make
  arraignments with you to make-up the in-class assignment.}
\item
  \emph{Course Engagement}: Students are expected to participate in the
  course by asking questions, providing thoughtful comments, and through
  making contributions to the group discussion portion of class.
  \textbf{Class discussion should be better than it would have been had
  you not attended.} Note that the professor has final say over what
  does or does not count as adequate participation.
\end{itemize}

\hypertarget{a-note-on-feedback}{%
\subsubsection{A Note on Feedback}\label{a-note-on-feedback}}

Each of your reflection papers will be graded in a holistic fashion. I
will examine and grade the document as whole, I am therefore not likely
to make specific comments on multiple aspects of each of your papers.
Any feedback will be provided on \href{https://lms.cofc.edu}{OAKS}. If
you adequately address what is asked for in each of the papers, you will
likely get all of most of the points. Any major issues with the
assignments or your writing in general (such as grammar) that results in
a significant loss of points (below 90\%) will be addressed with
feedback. Finally, I am happy to meet you with to discuss any individual
paper or your writing as a whole.

\hypertarget{late-work-policy}{%
\subsection{Late Work Policy}\label{late-work-policy}}

Late work is subject to a 48-hour grace period, and after that will be
penalized 10\% each day (24 hr period) it is late, up to 3 days. After 5
days beyond the due date the assignment will not be accepted unless you
have contacted me to let me know you need more time. For example, if an
assignment is due Thursday at 5:30 PM, the grace period ends on Saturday
at 5:30 PM and it is late as of 5:31 PM and you lose 10\%. After Sunday
at 5:31 PM you lose another 10\%, after Monday at 5:31 PM another 10\%,
and no work will be accepted after Tuesday at 5:30 PM.
